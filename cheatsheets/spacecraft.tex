\documentclass{cheatsheet}

\begin{document}

\section{Spacecraft Operations}

\begin{multicols}{3}
\begin{emphbox}
\textbf{Detecting something in sensor range}\\
\skillcheck{Electronics (sensors)}.

\textbf{Docking with another ship}\\
\skillcheck[Routine]{Pilot (spacecraft)}.

\textbf{Flying in an atmosphere}\\
Pilot \diemod{-2} (if partially streamlined and atmosphere 3+); or
\diemod{-4} (if unstreamlined) with \dice{1d} damage (ignoring armour)
per failure.

\textbf{Landing at a starport}\\
\skillcheck[Routine]{Pilot (spacecraft)}.

\textbf{Plotting a jump route}\\
\chrskillcheck[Easy]{edu}{Astrogation}, \diemod{-parsecs}.

\textbf{Making a jump}\\
\chrskillcheck[Easy]{edu}{Engineer (j-drive)}, task chain \textsc{dm}
from plotting, \diemod{-months} if behind on maintenance, \diemod{-2}
if using unrefined fuel, \diemod{-4} if within the hundred-diameter
limit.

\textbf{Unlocking a locked airlock}\\
\skillcheck[Very Difficult]{Electronics (computers)}.
\end{emphbox}

\subsection{Gas Giant Operations}

\textbf{Layer 1: Wisp}\\
Cannot skim fuel, offers no sensor interference, no real risk of
orbital decay.

\textbf{Layer 2: Extreme Shallow}\\
Can skim fuel at $\frac{1}{10}$ the normal rate.  Power loss results
in orbital decay in \dice{8d} hours.

\textbf{Layer 3: Shallow}\\
Can skim fuel at $\frac{1}{2}$ the normal rate.  Power loss results in
orbital decay in \dice{2d} hours.  Pilot checks have \diemod{-1}.

\textbf{Layer 4: Deep}\\
Can skim fuel at the normal rate.  Power loss results in orbital decay
in \dice{2d} minutes.  Pilot checks have \diemod{-2}.

\textbf{Layer 5: Extreme Deep}\\
Power loss results in \dice{2d} damage each round, and orbital decay
in \dice{2d} minutes.  Pilot checks have \diemod{-3}.

\textbf{Layer 6: Depths}\\
Without special protection, \dice{2d} damage each round.  Power loss
results in \dice{6d} damage each round, and orbital decay in \dice{2d}
minutes.  Pilot checks have \diemod{-4}.

\textbf{Layer 7: Abyssal Depths}\\
\dice{4d} damage each round.  Pilot checks have \diemod{-5}.  Sensors
do not work.

\textbf{Moving between layers:} \skillcheck{Pilot (spacecraft)},
\diemod{largest negative}.  On failure, the ship bounces off the other
layer and must succeed \skillcheck[Difficult]{Pilot (spacecraft)} or
take \dice{[layer]d} hull damage.

\textbf{Fuel skimming:} gain fuel equal to 1\% of hull tonnage per
pass, where a pass takes \dice{2d} minutes.

\textbf{Sensors:} checks to detect another vessel have \diemod{-2} for
every layer between them.

\textbf{Weapons:} Spacecraft weapons are less effective deeper in the
atmosphere.  See \textbf{Space Combat} section.

\subsection{Running Costs and Maintenance}

\begin{tabularx}{\linewidth}{lX} \toprule
  Item & Monthly Cost \\ \midrule
  Mortgage & varies \\
  Life Support & Cr1000/stateroom (Cr3000 if double occupancy); Cr100/low berth; Cr1000/person \\
  Fuel & Cr500/Rton; Cr100/URton \\
  Maintenance & 0.1\%/12 of purchase price \\ \bottomrule
\end{tabularx}

Maintenance must be done at a shipyard at least once per year.  If
maintenance is skipped, roll with \diemod{+months}, on 8+ roll for
critical hit:

\begin{tabularx}{\linewidth}{lX} \toprule
  \dice{2d} & Critical Hit and Outcome \\ \midrule
  2 to 4 & \textbf{Fuel Leak} \\
  & Lose \dice{1d $\times$ 10\%} capacity. \\
  5 to 7 & \textbf{Drive} \\
  & Roll \dice{1d}.  On 1 to 3 reduce Thrust by 1 and all Pilot checks are with Bane; on 4 to 6, J-Drive is disabled until repaired. \\
  8 to 9 & \textbf{Weapon Fault} \\
  & All checks with one random turrent (or weapon if no turrets) are with Bane. \\
  10 to 12 & \textbf{Power Plant} \\
  & Power reduced by 25\%; take a further \dice{1d} damage (ignoring armour); all crew take \dice{2d $\times 10$} rads/week. \\ \bottomrule
\end{tabularx}

\subsection{Jump Travel}

A jump uses 10\% of the hull tonnage for every parsec travelled and
takes \dice{148 + 6d} hours (about a week).

\textbf{Bad Astrogation}\\
Roll \dice{2d + Astrogator's Effect}.  The jump is bad on 5 or less.

\textbf{Bad Engineering}\\
Roll \dice{2d + Engineer's Effect}.  The jump is bad on 5 or less.

\textbf{Bad Jumps}\\
If one check is bad, all crew must make \textsc{end} and \textsc{int}
checks, one Routine the other Difficult.

If \chrcheck{end} is failed, the Traveller is nauseous and has
\diemod{-Effect} for \dice{2d} hours after entry and emergence.
Exceptional Failure incapacitates the Traveller for
\dice{2d$\times$30} minutes and imposes \diemod{-6} for \dice{4d}
hours.

If \chrcheck{int} is failed, the Traveller is irritable and has
\diemod{-2} for mental or interpersonal checks for the jump duration
and \dice{1d} hours after emergence.  Exception Failure causes a
significant mental breakdown.

\textbf{Very Bad Jumps}\\
If both checks are bad, or the ship misjumps, as \textbf{Bad Jumps}
with \diemod{-2} and also:

\begin{tabularx}{\linewidth}{lX} \toprule
  \dice{2d + dm} & Outcome \\ \midrule
  2 or less & No additional effects. \\
  3 to 5 & J-Drive needs \dice{2d} days recalibration. \\
  6 to 8 & J-Drive needs minor repairs. \\
  9 & J-Drive needs major repairs. \\
  10 to 12 & Intrusions occur. \\
  13 or more & Severe intrusions occur. \\ \bottomrule
\end{tabularx}

Use highest \textsc{dm} of:

\textbf{Bad Jump from both checks:} \diemod{-4}.\\
\textbf{Bad Jump from misjump:} \diemod{0}.\\
\textbf{Jumped within 100 diameters:} \diemod{+2}.\\
\textbf{Jumped within 10 diameters:} \diemod{+4}.

Intrusions destroy the portion of the ship they appear in, and also
destroy the J-Drive in all cases.  Destroys \dice{2d-2\%} hull
(\dice{2d+10\%} if severe) per day, requiring major repairs.

\textbf{Misjumps}\\
If \dice{Astrogator's Effect + Engineer's Effect} is 0 or less, as
\textbf{Bad Jumps} (or \textbf{Very Bad Jumps} if both checks failed)
and also:

\begin{tabularx}{\linewidth}{lX} \toprule
  \dice{2d} & Outcome \\ \midrule
  2 or less & Destroyed or lost in jumpspace \\
  3 to 4 & Jumps \dice{1d $\times$ 1d} parsecs in a random direction.  J-Drive destroyed. \\
  5 to 6 & Jumps \dice{2d} parsecs in a random direction.  J-Drive severely damaged. \\
  7 to 8 & Jumps \dice{1d} parsecs in a random direction. \\
  9 to 10 & Duration increased or decreased by \dice{1d} days.  J-Drive needs \dice{d3} days recalibration. \\
  11 to 12 & Emerge \dice{100 $\times$ 2d} diameters from target. \\ \bottomrule
\end{tabularx}

\subsection{Repairs}

\textbf{Critical Hits}\\
A Critical Hit can be temporarily repaired with \chrskillcheck{int
  {\normalfont or} edu}{Engineer}, with \diemod{-severity} and
\diemod{+attempts} but will break again after \dice{1d} hours.

Full repairs need a Engineer or Mechanic check (taking \dice{1d}
hours) and spare parts:

\begin{tabularx}{\linewidth}{Xr} \toprule
  Effect - Severity & Parts Required \\ \midrule
  1 or less & 1 ton \\
  2 & 0.8 tons \\
  3 & 0.3 tons \\
  4 & 0.4 tons \\
  5 & 0.2 tons \\
  6 or more & none \\ \bottomrule
\end{tabularx}

A destroyed weapon or piece of equipment cannot be repaired with spare
parts, and has to be replaced.

\textbf{Hull Damage}\\
Each lost hull point can be repaired with a
\chrskillcheck[Routine]{int {\normalfont or} edu}{Mechanic} (taking 1
hour), for 1 ton of spare parts per 10 hull points repaired.

\subsection{Common Travel Times}

\begin{tabularx}{\linewidth}{Xrr} \toprule
  Distance & at 1G & at 2G \\ \midrule
  \textbf{Surface to Orbit} & & \\
  \SI{10000}{\kilo\meter} & \SI{33}{\minute} & \SI{24}{\minute} \\
  \textbf{Orbiting Satellite} & & \\
  \SI{400000}{\kilo\meter} & \SI{211}{\minute} & \SI{149}{\minute} \\
  \textbf{Close Neighbour World} & & \\
  \SI{45000000}{\kilo\meter} & \SI{37.3}{\hour} & \SI{26.4}{\hour} \\
  \textbf{Far Neighbour World} & & \\
  \SI{255000000}{\kilo\meter} & \SI{88.7}{\hour} & \SI{62.7}{\hour} \\
  \textbf{Close Gas Giant} & & \\
  \SI{600000000}{\kilo\meter} & \SI{136.1}{\hour} & \SI{96.2}{\hour} \\
  \textbf{Far Gas Giant} & & \\
  \SI{900000000}{\kilo\meter} & \SI{166.7}{\hour} & \SI{117.9}{\hour} \\ \bottomrule
\end{tabularx}

See page 153 of the core rulebook for more.

\subsection{Security and Hacking}

See page 152 of the core rulebook.

\subsection{Sensors}

See page 150 of the core rulebook.
\end{multicols}

\section{Travellers In Space}

\begin{multicols}{3}
\subsection{Zero-G}
\textbf{Acclimatisation}\\
\diemod{-1} on all physical skill checks unless the Traveller has the
Athletics (dex) skill.

\textbf{Recoil}\\
Make an \skillcheck{Athletics (dex)} when making a melee attack or a
ranged attack with recoil, or automatically miss and spin out of
control.

Spinning can be stopped with an \skillcheck{Athletics (dex)}.

\subsection{Life Support}
A Traveller without life support takes \dice{1d} damage per minute.

\subsection{Vacuum}
A Traveller in vacuum takes a cumulative \dice{1d} damage per round,
ignoring armour, and \dice{2d$\times$10} rads per round.

\subsection{Radiation}
See page 77 of the core rulebook.
\end{multicols}

\section{Space Combat}

\begin{multicols}{3}
\begin{emphbox}
  \begin{center}
    \dice{2d + Pilot + Thrust + Tactics Effect}
  \end{center}

  \textbf{Tactics:} The commander of the ship (or fleet) may make a
  \skillcheck{Tactics (naval)}, adding the Effect to Initiative.

  \textbf{Surprise:} The surprised party skips their first round.

  \begin{center}
    \textbf{Round (6 minutes)}
  \end{center}

  Each ship \textbf{Manoeuvres}, then\\
  Each ship \textbf{Attacks}, then\\
  Each ship's crew take \textbf{Actions}.\\
  Ship's crews can take any number of \textbf{Reactions}.

  There are different rules for \textbf{dogfighting}.
\end{emphbox}

\subsection{Range Bands}

\begin{tabularx}{\linewidth}{Xrlr} \toprule
  Band & Min & Max & Thrust \\ \midrule
  Adjacent & --- & \SI{1}{\kilo\meter} & 1 \\
  Close & \SI{1}{\kilo\meter} & \SI{10}{\kilo\meter} & 1 \\
  Short & \SI{11}{\kilo\meter} & \SI{1250}{\kilo\meter} & 2 \\
  Medium & \SI{1251}{\kilo\meter} & \SI{10000}{\kilo\meter} & 5 \\
  Long & \SI{10001}{\kilo\meter} & \SI{25000}{\kilo\meter} & 10 \\
  Very Long & \SI{25001}{\kilo\meter} & \SI{50000}{\kilo\meter} & 25 \\
  Distance & \SI{50000}{\kilo\meter} & --- & 50 \\ \bottomrule
\end{tabularx}

\subsection{Manoeuvre Step}

\textbf{Move}\\
Spend Thrust (possibly over multiple rounds) to move up or down a
range band.  If both ships are moving, use relative Thrust.

If the ships enter Close or Adjacent range, switch to
\textbf{Dogfighting} rules.

\textbf{Aid Gunners}\\
Spend one point of Thrust and make a \skillcheck{Pilot (spacecraft)}
to start a Task Chain with the gunners.

\textbf{Dock}\\
Make a \skillcheck{Pilot (spacecraft)}.  If the other ship is
unwilling, make an opposed check with a Bane to the ship trying to
dock.

\subsection{Attack Step}

One Gunner attacks per turret and the Pilot attacks with fixed-mount
weapons.  Make a \chrskillcheck{dex}{Gunner (specialism)} with:

\begin{tabularx}{\linewidth}{Xr} \toprule
  Bonus & \textsc{dm} \\ \midrule
  Short Range & +1 \\
  Pulse Laser & +2 \\
  Beam Laser & +4 \\ \midrule
  Penalty & \\ \midrule
  Long Range & -2 \\
  Very Long Range & -4 \\
  Distant Range & -6 \\ \bottomrule
\end{tabularx}

\subsection{Action Step}

\textbf{Improve Initiative (Captain)}\\
Make a \skillcheck{Leadership}.  Add the Effect to Initiative for the
next round only.

\textbf{Jump (Engineer)}\\
Make both checks with \diemod{-2} due to bringing the timeframe down
to \dice{1d} minutes.

\textbf{Offline System (Engineer)}\\
Make a \chrskillcheck{edu}{Engineer (power)} to shut down any number
of systems.  Bringing any number of systems online requires a round.

\textbf{Overload Drive (Engineer)}\\
Make a \chrskillcheck[Difficult]{int}{Engineer (m-drive)} to increase
Thrust by 1 for the next round only.  Exceptional Failure inflicts a
\textbf{Critical Hit} with Severity 1.

This check has a cumulative \diemod{-2} on every attempt after the
first.  This penalty can be removed by making a \skillcheck{Engineer
  (m-drive)} in \dice{1d} hours.

\textbf{Overload Plant (Engineer)}\\
Make a \chrskillcheck[Difficult]{int}{Engineer (power)} to increase
Power by 10\% for the next round only.  Exceptional Failure indlicts a
\textbf{Critical Hit} with Severity 1.

This check has a cumulative \diemod{-2} on every attempt after the
first.  This penalty can be removed by making a \skillcheck{Engineer
  (power)} in \dice{1d} hours.

\textbf{Repair System (Engineer)}\\
See \textbf{Repairs} section.  Only \textbf{Critical Hits} may be
repaired, not hull damage.

\textbf{Reload Turret (Gunner)}\\
The turret cannot be used for an attack this round.

\textbf{Sensor Lock (Sensor Operator)}\\
Make a \skillcheck{Electronics (sensors)} to gain a Boon to attacks
against that target.

\textbf{Electronic Warfare (Sensor Operator)}\\
Make an opposed \skillcheck{Electronics (comms)} to jam
communications.

Or \skillcheck{Electronics (sensors)} to break a Sensor Lock.

Or \skillcheck[Difficult]{Electronics (sensors)} to destroy as many
missiles in a salvo as the Effect (once per round).

\textbf{Boarding Action (Marine)}\\
See \textbf{Boarding Actions} section.

\textbf{Reassignment}\\
Change to another duty, able to act as it from the next round.

\subsection{Reactions}

\textbf{Evasive Action (Pilot)}\\
Spend one Thrust to inflict the Pilot \textsc{dm} as a to-hit penalty
to one specific attack.

\textbf{Point Defence (Gunner)}\\
Make a \chrskillcheck{dex}{Gunner (turrent)}, with a laser turret,
right before a missile salvo hits, destroying as many missiles as the
Effect.  Gain \diemod{+1} from a double turret of lasers, and
\diemod{+2} from a triple turret of lasers.

A weapon used for Point Defence cannot also attack in the same round.

\textbf{Disperse Sand (Gunner)}\\
Make a \chrskillcheck{dex}{Gunner (turret)} and, on success, add
\dice{1d + Effect} to the ship's armour against a specific laser
attack.

This may also be used to attack a boarding party, dealing \dice{8d}
points of damage to each member of the party.

\subsection{Missile Combat}

\begin{tabularx}{\linewidth}{Xr} \toprule
  Range & Rounds to impact \\ \midrule
  Medium and below & 0 \\
  Long & 1 \\
  Very Long & 4 \\
  Distant & 10 \\ \bottomrule
\end{tabularx}

For the purposes of evasive action, a missile salvo has an effective
Thrust of 10.  If a salvo hasn't hit in 10 rounds, it becomes inert.

\textbf{Smart}\\
If launched below Short range, any Smart trait is lost.

\textbf{To-Hit Roll}\\
Roll \dice{2d + Num Missiles}, plus any other modifiers (like Smart).
A salvo launched at Distant range suffers \diemod{-6}.

\textbf{Damage Roll}\\
Roll for damage as if it were a single missile and deduct armour, but
multiply by the Effect of the to-hit roll (rather than add).

\subsection{Damage}

Add to-hit Effect to weapon damage.  Armour reduces damage.  Unlike
normal combat, an Effect of 6 or more does \emph{not} deal guaranteed
damage.

The target suffers a \textbf{Critical Hit} every 10\% (rounded up) of
hull it loses at Severity 1, and for every to-hit roll with an Effect
of 6 which inflicts damage.

\textbf{Double and Triple Turrets}\\
If a turret has different types of weapons, only one type may be used
to attack a target in a round.  If the weapons are the same type, they
can all attack, adding \diemod{+1} to the damage per additional damage
die.

Sandcasters can be linked in the same way to more effectively block
laser attacks.

\textbf{Damage Scale}\\
A spacecraft weapon attacking a ground target suffers \diemod{-2}
to-hit but $\times$10 damage, with Blast 10.

A ground weapon attacking a spacecraft target gains \diemod{+2} to-hit
but $\div$10 damage.

Scaling is done after any other modifiers.

\subsection{Critical Hits}

\begin{tabularx}{\linewidth}{lX} \toprule
\dice{2d} & Hit Location \\ \midrule
2 & Sensors \\
3 & Power Plant \\
4 & Fuel \\
5 & Weapon \\
6 & Armour \\
7 & Hull \\
8 & M-Drive \\
9 & Cargo \\
10 & J-Drive \\
11 & Crew \\
12 & Computer \\ \bottomrule
\end{tabularx}

See page 159 of the core rulebook for Critical Hit effects.

Any extra damage caused by the Critical Hit effect ignores armour.

\textbf{Severity}\\
The Severity of the hit is the damage which inflicted it, divided by
10, rounded up.  If the location has already taken a Critical Hit, use
the Severity of the current hit or one plus the Severity of the
previous hit (whichever is greater).

Once a location has reached Severity 6, further Critical Hits inflict
\dice{6d} extra damage.

\subsection{Weapon Power Requirements}

\begin{tabularx}{\linewidth}{Xr} \toprule
  Weapon & Power Required \\ \midrule
  Beam Laser & 4 \\
  Missile Rack & 0 \\
  Particle Beam & 8 \\
  Pulse Laser & 4 \\
  Sandcaster & 0 \\
  Turret & 1 \\ \bottomrule
\end{tabularx}

\subsection{Close Range Combat (Dogfighting)}

Combat rounds are 6 seconds long.  Ships of 100 tons or more suffer
\diemod{-6} on all to-hit rolls.

Each round, make an opposed \skillcheck{Pilot}, with:

\begin{tabularx}{\linewidth}{lX} \toprule
  Dogfighter & \textsc{dm} \\ \midrule
  Ship is 50 tons or more & -1 \\
  Ship is 100 tons or more & -2 \\
  For each extra 100 tons & -1 \\
  For each extra enemy & -1 \\
  Ship's thrust & +1 per point dedicated to dogfighting \\ \bottomrule
\end{tabularx}

\textbf{On draw:} neither ship may attack the other with fixed weapons.

\textbf{On success:} the winner chooses the relative position of the
ships and gains \diemod{+2} for to-hit rolls, whereas the loser
suffers \diemod{-2}.  The winner uses the difference as a positive
\textsc{dm} in the next round's opposed \skillcheck{Pilot}.

\subsection{Boarding Actions}

See page 163 of the core rulebook and page 17 of Pirates of Drinax.

\subsection{Gas Giant Operations}

See page 157 of the Traveller Companion
\end{multicols}

\end{document}
